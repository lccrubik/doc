\documentclass[11pt,a4paper]{article}
\usepackage[utf8]{inputenc}
\usepackage{fullpage} % esto hace la página mas ancha, queda mejor?
\usepackage{verbatimbox}

\begin{document}

\title{TP EDyA 1 - Cubo de Rubik}

\maketitle

\abstract{La idea de este trabajo es realizar un programa que resuelva un cubo de Rubik, de manera general, representando el problema como una búsqueda. El programa recibirá por su entrada estándar un cubo de Rubik y deberá imprimir por su salida estándar la secuencia de movimientos que lo resuelven (no necesariamente la secuencia óptima).}

\section{Introducción}

Para después.

\section{Comportamiento del programa}

\subsection{Entrada}
El programa recibirá una cadena de 54 números representando al cubo, donde los números del 1 al 6 representan un color del cubo. Por ejemplo, el siguiente es un cubo resuelto: \\
\begin{center}
\begin{verbbox}
      1 1 1 
      1 1 1 
      1 1 1 
2 2 2 3 3 3 4 4 4 5 5 5 
2 2 2 3 3 3 4 4 4 5 5 5 
2 2 2 3 3 3 4 4 4 5 5 5
      6 6 6 
      6 6 6 
      6 6 6 
\end{verbbox}

\theverbbox
\end{center}

El cubo se representa en la entrada por su 'desdoblamiento'. No se debe confiar en el espaciado de la entrada. Sólo se debe asumir que se darán como entrada 54 números representando a los colores de las caras, en el orden anterior. Es decir, la entrada anterior es equivalente a:

\begin{center}
\begin{verbbox}
1 1 1 1 1 1 1 1 1 2 2 2 3 3 3 4 4 4 5 5 5 2 2 2 3 3 3
4 4 4 5 5 5 2 2 2 3 3 3 4 4 4 5 5 5 6 6 6 6 6 6 6 6 6 
\end{verbbox}

\theverbbox
\end{center}

Usando el ejemplo anterior, le damos un nombre a cada cara según la tabla. \\
% Haciendo referencia a las caras por el número que contienen, la cara 1 representa la cara superior, y etc, según la tabla. \\

\begin{center}
\begin{tabular}{| c || c | c |}
\hline 
Número & Nombre & Código \\ \hline
1 & Superior (Upper) & U \\ \hline
2 & Izquierda (Left) & L \\ \hline
3 & Frente (Front) & F \\ \hline
4 & Derecha (Right) & R \\ \hline
5 & Atrás (Back) & B \\ \hline
6 & Abajo (Down) & D \\ \hline
\end{tabular}
\end{center}


Notar que no es necesario que el cubo resuelto quede exactamente como el ejemplo anterior. Cualquier cubo está resuelto mientras todos los colores (números) de cada cara sean iguales. 
De hecho, dado un cubo de entrada, los números centrales de cada cara nunca cambian (¿por qué?), por lo que sería imposible resolverlo. % ponemos esto o no? agregamos las rotaciones o no? si ponemos rotaciones estamos mintiendo con lo segundo


\end{document}
